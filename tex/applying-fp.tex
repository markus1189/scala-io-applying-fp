\documentclass{beamer}

\usepackage{minted}
\usepackage{pgf}
\usepackage[utf8]{inputenc}
\usepackage{mdframed}

\BeforeBeginEnvironment{minted}{\begin{mdframed}}
\AfterEndEnvironment{minted}{\end{mdframed}}

\title{Applying \textbf{one} FP Pattern: Monoids}
\subtitle{Original Title: Applying FP Patterns (sorry)}
\author{Markus Hauck}
\date{Scala.IO, 2016}
\subject{Computer Science}

\input{./beamer-setup}

\begin{document}
\frame{\titlepage}

\section{Intro}
\label{sec:intro}

\begin{frame}
  \frametitle{Introduction}
  \begin{itemize}
  \item FP has become very popular
  \item Even without complete commitment, useful patterns
  \item Today we will have a look at one of them: Monoids
  \end{itemize}
\end{frame}

\section{Monoids}
\label{sec:monoids}

\begin{frame}
  \frametitle{Lego vs Duplo}
  FP = Lego, OO = Duplo?
\end{frame}

\begin{frame}
  \frametitle{Monoids}
  \begin{itemize}
  \item intuition: ``summon \& combine stuff''
  \item you can create values from thin air
  \item combine two values into one
  \end{itemize}
\end{frame}

\begin{frame}
  \frametitle{I thought this was about ``applying''}
  \begin{itemize}
  \item damn right, so let's do that!
  \end{itemize}
\end{frame}

\begin{frame}
  \frametitle{Apache Spark}
  \begin{itemize}
  \item Apache Spark:
    \begin{itemize}
    \item analysis of a text (huuuuge)
    \item run in cluster
    \end{itemize}
  \item some possible metrics over text
    \begin{itemize}
    \item word count
    \item char count
    \item min/max word length
    \item avg word length
    \item \dots (be flexible)
    \end{itemize}
  \item \textbf{goal}: single traversal $\leftrightarrow$ \textbf{easy} composition
  \end{itemize}
\end{frame}

\begin{frame}[fragile]
  \frametitle{RDDs and Folds}
\begin{minted}{scala}
abstract class RDD[T] {
  /** Aggregate the elements of each partition,
    * and then the results for all the partitions,
    * using a given associative function and a
    *  neutral "zero value".
    */
  def fold(zeroValue: T)(op: (T, T) => T): T
}
\end{minted}

  \begin{itemize}
  \item \textbf{associative} and \textbf{neutral value} --- rings a bell?
  \end{itemize}
\end{frame}

\begin{frame}[fragile]
  \frametitle{Monoidal RDDs}
\begin{minted}{scala}
implicit class MonoidRDD[T](val rdd: RDD[T])
  extends AnyVal {

  // avoid conflicts with fold/reduce etc
  def combine(implicit M: Monoid[T]): T =
    rdd.fold(M.empty)(M.combine(_,_))

}
\end{minted}
\end{frame}

\begin{frame}[fragile]
  \frametitle{Using Monoids}
  \begin{itemize}
  \item monoids are perfect for this
  \item all we need from Spark is the \texttt{fold} function on \texttt{RDD}s
  \end{itemize}
\begin{minted}{scala}
foo bar baz
\end{minted}
\end{frame}

% IDEA: use the `animate' package to show animations

\begin{frame}
  \frametitle{More Monoid Tricks}
  \begin{itemize}
  \item ``filter'' values via \texttt{mempty} value
  \item map + reduce == two phase computation via monoids
  \item finger trees, implement multiple data structures using monoids
  \item drawing diagrams (haskell)
  \end{itemize}
\end{frame}

\end{document}
