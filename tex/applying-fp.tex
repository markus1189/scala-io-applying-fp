\documentclass{beamer}

\usepackage{minted}
\usepackage{pgf}
\usepackage[utf8]{inputenc}
\usepackage{mdframed}

\BeforeBeginEnvironment{minted}{\begin{mdframed}}
\AfterEndEnvironment{minted}{\end{mdframed}}

\title{Applying \textbf{one} FP Pattern: Monoids}
\subtitle{Original Title: Applying FP Patterns (sorry)}
\author{Markus Hauck}
\date{Scala.IO, 2016}
\subject{Computer Science}

\useinnertheme{default}

\definecolor{MidnightBlue}{RGB}{68,72,169}
\definecolor{Supernova}{RGB}{227,158,0}
\definecolor{MediumSpringGreen}{RGB}{0, 227, 158}
\definecolor{Purple}{RGB}{158,0,227}
\definecolor{ScalaIO}{RGB}{187,0,39}

\definecolor{shadecolor}{RGB}{227,158,0}

\usecolortheme[named=ScalaIO]{structure}

\setbeamercolor{my header}{fg=ScalaIO,bg=black}
\setbeamercolor{my footer}{fg=ScalaIO!90,bg=black}
\setbeamercolor{my footer 2}{fg=ScalaIO!90,bg=black!90}
\setbeamercolor{my page number}{fg=black,bg=ScalaIO!90}

\setbeamercolor{section in head/foot}{fg=white}
\setbeamercolor{mini frame}{fg=ScalaIO}

\setbeamertemplate{headline}{%
  \begin{beamercolorbox}[wd=\paperwidth,ht=5ex,dp=5ex]{my header}
    \insertnavigation{\paperwidth}
  \end{beamercolorbox}
  \begin{pgfpicture}{0mm}{0mm}{0mm}{0mm}
    \pgfsetlinewidth{0.5mm}
    \color{ScalaIO}
    \pgfline{\pgfpoint{0.01\paperwidth}{-1mm}}{\pgfpoint{0.8\paperwidth}{-1mm}}
  \end{pgfpicture}
  \begin{pgfpicture}{0mm}{0mm}{0mm}{0mm}
    \pgfsetlinewidth{0.5mm}
    \color{ScalaIO}
    \pgfline{\pgfpoint{0.0025\paperwidth}{-0.76mm}}{\pgfpoint{0.0025\paperwidth}{-0.07\paperwidth}}
  \end{pgfpicture}
}

\setbeamertemplate{footline}{%
  \hbox{%
  \begin{beamercolorbox}[wd=0.55\paperwidth,ht=2.5ex,dp=1ex]{my footer}
    \hspace{1mm}\insertshortauthor{} - \insertshorttitle{} - codecentric AG
  \end{beamercolorbox}
  \hspace{-2mm}
  \begin{beamercolorbox}[wd=0.40\paperwidth,ht=2.5ex,dp=1ex]{my footer 2}
    \hspace{1mm}\insertsubsection{}
  \end{beamercolorbox}
  \hspace{-2mm}
  \begin{beamercolorbox}[wd=0.06\paperwidth,ht=2.548ex,dp=1.01ex,center]{my page number}
    \insertframenumber{}
  \end{beamercolorbox}
}
  \begin{pgfpicture}{0mm}{0mm}{0mm}{0mm}
    \pgfsetlinewidth{0.5mm}
    \color{ScalaIO}
    \pgfline{\pgfpoint{0.8\paperwidth}{4.5mm}}{\pgfpoint{0.99\paperwidth}{4.5mm}}
  \end{pgfpicture}
  \begin{pgfpicture}{0mm}{0mm}{0mm}{0mm}
    \pgfsetlinewidth{0.5mm}
    \color{ScalaIO}
    \pgfline{\pgfpoint{0.985\paperwidth}{4.255mm}}{\pgfpoint{0.985\paperwidth}{0.1\paperwidth}}
  \end{pgfpicture}
}

\setbeamertemplate{navigation symbols}{}


\setbeamercolor*{block title}{fg=white,bg=black}
\setbeamercolor*{block body}{bg=black!20}

\setbeamercolor*{block title alerted}{use={normal text,alerted text},fg=black,bg=ScalaIO}
\setbeamercolor*{block body alerted}{bg=black,fg=black!20}

\setbeamercolor*{block title example}{fg=black,bg=ScalaIO!90}
\setbeamercolor*{block body example}{bg=black!20}
\setbeamercolor*{example text}{fg=ScalaIO}

\setbeamertemplate{mini frames}[box]
\setbeamersize{mini frame size=3pt}

\setbeamertemplate{blocks}[rounded][shadow=true]

\begin{document}
\frame{\titlepage}

\section{Intro}
\label{sec:intro}

\begin{frame}
  \frametitle{Introduction}
  \begin{itemize}
  \item FP has become very popular
  \item Even without complete commitment, useful patterns
  \item Today we will have a look at one of them: Monoids
  \end{itemize}
\end{frame}

\section{Composition}

\begin{frame}
  \frametitle{Lego vs Duplo}
  FP = Lego, OO = Duplo?
\end{frame}

\section{Monoids}

\begin{frame}
  \frametitle{Monoids}
  \begin{itemize}
  \item intuition: ``summon \& combine stuff''
  \item you can create values from thin air
  \item combine two values into one
  \end{itemize}
\end{frame}

\section{Apache Spark}

\begin{frame}
  \frametitle{I thought this was about ``applying''}
  \begin{itemize}
  \item damn right, so let's do that!
  \end{itemize}
\end{frame}

\begin{frame}
  \frametitle{Apache Spark}
  \begin{itemize}
  \item Apache Spark:
    \begin{itemize}
    \item analysis of a text (huuuuge)
    \item run in cluster
    \end{itemize}
  \item some possible metrics over text
    \begin{itemize}
    \item word count
    \item char count
    \item min/max word length
    \item avg word length
    \item \dots (be flexible)
    \end{itemize}
  \item \textbf{goal}: single traversal $\leftrightarrow$ \textbf{easy} composition
  \end{itemize}
\end{frame}

\begin{frame}[fragile]
  \frametitle{RDDs and Folds}
\begin{minted}{scala}
abstract class RDD[T] {
  /** Aggregate the elements of each partition,
    * and then the results for all the partitions,
    * using a given associative function and a
    *  neutral "zero value".
    */
  def fold(zeroValue: T)(op: (T, T) => T): T
}
\end{minted}

  \begin{itemize}
  \item \textbf{associative} and \textbf{neutral value} --- rings a bell?
  \end{itemize}
\end{frame}

\begin{frame}[fragile]
  \frametitle{Monoidal RDDs}
\begin{minted}{scala}
implicit class MonoidRDD[T](val rdd: RDD[T])
  extends AnyVal {

  // avoid conflicts with fold/reduce etc
  def combine(implicit M: Monoid[T]): T =
    rdd.fold(M.empty)(M.combine(_,_))

}
\end{minted}
\end{frame}

\begin{frame}[fragile]
  \frametitle{Using Monoids}
  \begin{itemize}
  \item monoids are perfect for this
  \item all we need from Spark is the \texttt{fold} function on \texttt{RDD}s
  \end{itemize}
\begin{minted}{scala}
foo bar baz
\end{minted}
\end{frame}

% IDEA: use the `animate' package to show animations

\section{Almost Done}

\begin{frame}
  \frametitle{More Monoid Tricks}
  \begin{itemize}
  \item ``filter'' values via \texttt{mempty} value
  \item map + reduce == two phase computation via monoids
  \item finger trees, implement multiple data structures using monoids
  \item drawing diagrams (haskell)
  \end{itemize}
\end{frame}

\end{document}
